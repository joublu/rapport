\subsection{Contexte du stage}

Dans un contexte de transition écologique, la mobilité douce est de plus en plus mise en avant. Le vélo est un moyen de transport qui permet de réduire les émissions de gaz à effet de serre, et est en plus bénéfique pour la santé, mais il est encore peu utilisé dans les villes françaises. Il est donc important de le rendre accessible à tous les usagers, quel que soit leur niveau d'expérience. Cependant, les infrastructures cyclables sont souvent inadaptées aux besoins des usagers, ce qui peut décourager leur utilisation.

Mon stage s'insère dans le projet Amélioration des Réseaux cyclaBles pour l'équIté TerRitoriale en AménaGemEnt (ARBITRAGE). L'objectif de ce projet est de tirer parti d'une collaboration pluridisciplinaire afin d'étudier comment améliorer le maillage et la qualité d'un réseau cyclable pour garantir l'équité en termes d'accès aux services (emploi, commerces, soins, loisirs, ...) en prenant en compte la demande de mobilité existante. Sur le long terme, l'ambition est de développer un outil d'aide à la décision permettant aux collectivités de prioriser des projets d'amélioration de voies cyclables (modernisation du réseau existant ou aménagement de nouvelles infrastructures). 

La demande de financement a été adressée à l'université de Tours, dans le cadre d'un appel à projet ART (Actions de recherches transversales). L'objectif de cet appel est de soutenir des projets interdisciplinaires de chercheurs de l'université.

Le financement du projet couvre deux volets : deux stages pour travailler à la définition et à la construction d'indicateurs d'accessibilité via un réseau cyclable à partir de données ouvertes (Lou-Ann Deniau et moi-même), ainsi qu'un séminaire ouvert aux chercheurs de l'Université de Tours sur la mobilité à vélo, de sorte à fédérer des recherches transversales.

J'ai repris le travail de Tifenn Rault et d'Emmanuel Néron, chercheurs au LIFAT, ainsi que de Jérôme Lécuyer et d'Alaâ Chakori Semmane, deux stagiaires au LIFAT en 2024 qui ont créé une application de construction et de visualisation d'indicateurs d'accessibilité pour les voies cyclables. Mon stage s'insère dans un projet en collaboration avec des chercheurs du laboratoire UMR CNRS 7324 - CItés, TERritoires, Environnement, Sociétés (CITERES) et de la Maison des Sciences de l'Homme - Val de Loire, une structure fédérative d'équipes de Recherche en Sciences Humaines et Sociales. 

\subsection{Présentation de l'entreprise}

Le Laboratoire d'Informatique Fondamentale et Appliquée de Tours (LIFAT) \cite{lifat} regroupe des chercheurs et enseignants-chercheurs de l'Université de Tours et de l'INSA Centre Val de Loire. En 2025, il compte 47 enseignants-chercheurs (professeurs, maîtres de conférences), 31 doctorants et 10 post-doctorants.

Le laboratoire est donc spécialisé en Sciences des Données, avec des spécialités reconnues dans les domaines de l'Intelligence Artificielle et de l'Optimisation.

Le LIFAT a une expérience en matière de collaborations académiques (au niveau national et international) et de partenariats industriels. Les nombreuses opportunités de transfert technologique (vers le monde social et économique) des recherches menées dans le laboratoire ont donné lieu à la création d'un Centre d'Expertise et de Transfert Universitaire (CETU) au laboratoire appelé ILIAD.

Le laboratoire est actuellement organisé en trois équipes de recherche :

\begin{itemize}
    \item Bases de données et Traitement du langage naturel (BdTln),
    \item Reconnaissance des Formes et Analyse d'Images (RFAI),
    \item Recherche Opérationnelle, Ordonnancement et Transport (ROOT). J'étais rattaché à cette dernière lors de mon stage.
\end{itemize}

J'ai intégré l'équipe ROOT, spécialisée dans l'optimisation combinatoire et la modélisation mathématique. Elle développe des méthodes et des outils pour résoudre des problèmes complexes d'optimisation dans divers domaines d'application, tels que les mobilités urbaines, la santé numérique et l'industrie 4.0.

\subsection{Objectifs du stage}

L'objectif de mon travail est donc de trouver des méthodes permettant d'optimiser un réseau cyclable en fonction d'indicateurs d'équités choisis. En fonction d'un budget fixé, comment optimiser l'équité d'un réseau cyclable ?
