\subsection{Bilan du stage}

\subsection{Développements ultérieurs}

Il est possible d'ajouter davantage de paramètres pour gagner en réalisme. 

Pour le modèle d'optimisation sur le nombre distinct de catégories, il serait par exemple possible de prendre en compte la population de chaque carreau, pour augmenter la priorité de chaque PCC à améliorer en fonction de la population de son carreau de départ.

J'ai pris en compte cette amélioration potentielle dans mes programmes, de sorte qu'elle soit facilement implémentable par la suite.

Il est aussi possible de faire passer en premier les carreaux ayant une population défavorisée.

Une autre amélioration possible serait d'implémenter une optimisation sur les indicateurs créés par Lou-Ann Deniau, 

\subsection{Apport personnel}

Stage pluridisciplinaire, avec des aspects de recherche opérationnelle, de développement informatique et de visualisation de données.

Travail avec des Géographe - cartographe

Travail d'équipe, vulgarisation, expliquer à des personnes qui ne s'y connaissent pas en informatique.

%----

Projet professionnel