Je tiens tout d'abord à remercier le LIFAT et Monsieur Hubert Cardot, son directeur, qui, en m'accueillant au sein du laboratoire, m'ont permis d'avoir une expérience très enrichissante dans le domaine de la recherche en informatique. 

Je tiens également à exprimer ma gratitude à Madame Tifenn Rault, maîtresse de conférences à Polytech Tours et membre du LIFAT, ma tutrice de stage, pour sa confiance, ses conseils et pour son aide. Je lui suis particulièrement reconnaissant pour la qualité de son encadrement.

Je souhaite remercier Monsieur Emmanuel Néron, professeur des universités en informatique à Polytech Tours et membre du LIFAT, qui a également pu me guider dans mes travaux.

Je souhaite remercier Mesdames Sandra Hemon et Annie Simon, secrétaires au LIFAT, pour leur disponibilité et pour leur aide administrative.

Je remercie Messieurs Dominique Andrieu, Hervé Baptiste, Laurent Cailly, cartographes et enseignants chercheurs en aménagement urbanisme, ainsi que Lou-Ann~Deniau, en stage auprès d'eux, pour leur collaboration sur le projet et pour leurs conseils.

Je remercie aussi les doctorants du LIFAT, pour leurs conseils, leurs idées et leur soutien. Mon expérience au LIFAT n'aurait pas été si agréable sans eux. Je remercie en outre tous les membres du LIFAT avec qui j'ai pu échanger. 

Je remercie l'université de Tours pour le financement de mon stage, et je remercie le Mésocentre de Calcul CaSciModOT pour m'avoir laissé effectuer mes calculs sur leurs serveurs.

Je tiens finalement à remercier l'INSA de Rouen et le département Génie Mathématique, l'accompagnement dont j'ai bénéficié m'a permis de trouver un stage adhérant à mon projet professionnel, et dont la qualité de la formation m'a permis de contribuer à un projet de recherche. Je remercie aussi Monsieur Arnaud Knippel, mon tuteur académique.

% % % Je remercie Luca Zepponi, à qui j'ai volé plein de choses pour avoir un beau rapport \LaTeX.