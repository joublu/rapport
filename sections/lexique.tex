% \begin{itemize}
%     \item Carreau : Carreau Filosofi, délimitant un territoire à des fins de statistiques sur sa population.
%     \item LTS : \emph{Level of Traffic Stress}, niveau de stress d'un tronçon de voie pour les cyclistes.
%     \item POI : \emph{Point Of Interest}.
%     \item PPOI : \emph{Potential Point Of Interest}, POI qui est potentiellement accessible par le réseau cyclable.
%     \item Atteint : se dit d'un POI auquel on peut accéder en empruntant seulement des voies inférieures à facteur LTS choisi.
%     \item Potentiellement accessible : se dit d'un POI qui serait atteint si le réseau était entièrement sécurisé.
%     \item Noeud : Intersection entre deux tronçons du réseau cyclable.
%     \item Arc : Tronçon du réseau cyclable, reliant deux noeuds.
%     \item Graphe : Ensemble de noeuds et d'arcs, modélisant le réseau cyclable.
%     \item Noeud délégué : Noeud au centre d'un carreau (Filosofi).
%     \item PM : \emph{Performance Measure}, indicateur de performance.
%     \item Indicateur d'équité : Un indicateur est un critère de performance, croisant des données sur le graphe et des données socio-démographiques.
%     \item Visibilité exacte : en opposition à visibilité PCC, voir section \ref{sect:2visi}.  
%     \item Modèle exact : modèle d'optimisation trouvant une des solutions optimales. Un modèle exact peut l'être sur la visibilité exacte (modèle exact sur la visibilité exacte), ou sur la visibilité PCC (modèle exact sur la visibilité PCC).
%     \item OV : \emph{Objective Value}, valeur objectif.
%     \item Sécurisé : se dit d'un tronçon que l'on considère comme sûr pour les cyclistes, c'est à dire de niveau LTS inférieur ou égal à un certain niveau.
% \end{itemize}

\begin{itemize}
    \item Arc : Tronçon du réseau cyclable, reliant deux noeuds.
    \item Atteint : se dit d'un POI auquel on peut accéder en empruntant seulement des voies inférieures à facteur LTS choisi.
    \item Carreau : Carreau Filosofi, délimitant un territoire à des fins de statistiques sur sa population.
    \item Graphe : Ensemble de noeuds et d'arcs, modélisant le réseau cyclable.
    \item Indicateur d'équité : Un indicateur est un critère de performance, croisant des données sur le graphe et des données socio-démographiques.
    \item LTS : \emph{Level of Traffic Stress}, niveau de stress d'un arc de voie pour les cyclistes.
    \item Modèle exact : modèle d'optimisation trouvant une des solutions optimales du problème. Un modèle exact peut l'être sur la visibilité exacte (modèle exact sur la visibilité exacte), ou sur la visibilité PCC (modèle exact sur la visibilité PCC).
    \item Noeud : Intersection entre deux arcs du réseau cyclable.
    \item Noeud délégué : Noeud au centre d'un carreau (Filosofi).
    \item OV : \emph{Objective Value}, valeur objectif.
    \item PM : \emph{Performance Measure}, indicateur de performance.
    \item POI : \emph{Point Of Interest}.
    \item Potentiellement accessible : se dit d'un POI qui serait atteint si le réseau était entièrement sécurisé.
    \item PPOI : \emph{Potential Point Of Interest}, POI qui est potentiellement accessible par le réseau cyclable.
    \item Sécurisé : se dit d'un arc que l'on considère comme sûr pour les cyclistes, c'est à dire de niveau LTS inférieur ou égal à un certain niveau.
    \item Visibilité exacte : en opposition à visibilité PCC, voir section \ref{sect:2visi}.  
\end{itemize}

